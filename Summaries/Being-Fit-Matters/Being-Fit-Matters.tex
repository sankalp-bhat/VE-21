\documentclass[a4paper, 12pt]{extarticle}
\usepackage[a4paper, total={6in, 10in}]{geometry}
\usepackage[english]{babel}
\usepackage[utf8x]{inputenc}
\usepackage{amsmath}
\usepackage{graphicx}
\usepackage{indentfirst}

\title{Value Education - 1\\
Being Fit Matters
}

\author{Sankalp S Bhat\\
2020112018
}
\begin{document}
\maketitle
\thispagestyle{empty}

\section*{About the Session}
This session emphasized the importance of being fit, about its benefits, and practical ways to keep oneself fit. In the digital age, maintaining appropriate fitness becomes a very difficult task when most people are glued to their screens for a significant portion of the day. To remedy this, we were told about various ways to keep ourselves active and maintain a minimum level of fitness.

Dr. K S Kamalakar presented to us a series of clips about the physical limitations of our bodies, and encouraged us to exercise daily to remain fit. He also explained about the gradually increasing demands of the muscles, and what happens when they are not satisfied.
      
\section*{What I liked about the session}

What I really liked about the session was the various ways that were told to us by which we could keep ourselves fit, along with the multitude of different fitness routines.
    
I am also in agreement with the effective utilisation of power naps to rest yourself, and liked how the speaker mentioned about Yoga and the impact it can have on our physical as well as mental health.
     
 \section*{What I disliked about the session}
 
I was displeased about the inconsideration of the impact that the COVID-19 pandemic has had on our daily lives, and how it has made exercising difficult for a lot of us. Adding to that the hectic workload at college, reducing screen-time is not a realistic option for most students.

 \section*{Takeaways}
    
Today's session has made me realise how neglectful I have been regarding my physical fitness, and has also opened my eyes to how important good food, decent sleep and exercising are to one's physical fitness.

\end{document}

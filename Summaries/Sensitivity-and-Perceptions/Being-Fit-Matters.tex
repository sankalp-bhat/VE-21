\documentclass[a4paper, 12pt]{extarticle}
\usepackage[a4paper, total={6in, 10in}]{geometry}
\usepackage[english]{babel}
\usepackage[utf8x]{inputenc}
\usepackage{amsmath}
\usepackage{graphicx}
\usepackage{indentfirst}

\title{Value Education - 1\\
Sensitivity and Perceptions}

\author{Sankalp S Bhat\\
2020112018
}
\begin{document}
\maketitle
\thispagestyle{empty}

\section*{About the Session}
This session was about the various misconceptions and biases that people hold about each other. We also talked about various forms of media and how they help shape some of our notions about various sections of society. Racial and gender discrimination were also brought to light.

Dr. Kavita Vemuri presented to us a series of images about the pre-conceived biases that we hold, including examples about fairness creams and the unhealthy beauty standards that they set for young women, the India Pakistan war and how citizens from each side of the border harbour unnecessary hate towards one another, and gender stereotypes that we come across on a daily basis.
      
\section*{What I liked about the session}

What I really liked about the session was how the speaker addressed various important topics pertaining to discrimination, and how it was related to the evils we see everyday in life.
    
I am also in agreement with the how the African American community is viewed with a negative bias, and how years of discrimination against the blacks has led to them being perceived as delinquents and lawbreakers.
     
 \section*{What I disliked about the session}
 
I was displeased about the lack of methods being told to us to correct ourselves, and also deal with issues plaguing the youth of today such as abnormal beauty standards and addiction to narcotics.

Something that could also have been touched upon more was how to love yourself more, and not let yourself be affected by grudges or biases that people could hold against you.
 \section*{Takeaways}
    
I realise how neglectful I have been towards existing inequalities, and have realised how I can correct myself and not hold grudges/biases based on preconceived notions.

\end{document}

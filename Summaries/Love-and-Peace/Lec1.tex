\documentclass[a4paper]{article}

\usepackage[english]{babel}
\usepackage[utf8x]{inputenc}
\usepackage{amsmath}
\usepackage{graphicx}

\title{Value Education - 1\\
Love and Peace - A musical journey
}

\author{Sankalp S Bhat\\
2020112018
}
\begin{document}
\maketitle
\thispagestyle{empty}

\section*{About the Session}

      This session portrayed the hopeful realisation of eventual harmony between India and Pakistan through the medium of border parades, music, and interviews with citizens, focusing on music as the most important medium of the documentary. 
      
      Professor Harjinder Singh presented to us a series of clips, shot across India and Pakistan, showcasing the diversity of religions, cultures and styles of music. 
      
      Another important part of the session was the portrayal of the turbulent relationship between India and Pakistan, as expressed by the various interviews shown throughout the documentary, and the various negative stereotypes, both positive and negative, that the interviewees held about citizens of the neighbouring country.

\section*{What I liked about the session}

    I really enjoyed the music showcased throughout the session, and have checked out more of the same.
    
    I also empathised with the hopeful citizens of either country, and loved their attempts at trying to spread love and peace through the medium of art, in their own special way.
     
 \section*{What I disliked about the session}
 
    I am displeased about the lack of new ideas and issues that could have been put forward, and the lack of depth of discussion in terms of the viewpoints showcased in the session, and being shown the same idea of India-Pakistan unity without looking into the real world issues between the countries.
 
 \section*{Takeaways}
    
    I realised the impact that artists and their work can have on society, and of the fundamental love that binds us as humans, above all geopolitical borders.

\end{document}
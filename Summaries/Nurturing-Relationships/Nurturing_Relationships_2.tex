\documentclass[a4paper, 12pt]{extarticle}
\usepackage[a4paper, total={6in, 10in}]{geometry}
\usepackage[english]{babel}
\usepackage[utf8x]{inputenc}
\usepackage{amsmath}
\usepackage{graphicx}
\usepackage{indentfirst}

\title{Value Education - 1\\
Nurturing Relationships - 2}

\author{Sankalp S Bhat\\
2020112018
}
\begin{document}
\maketitle
\thispagestyle{empty}

\section*{About the Session}
This session was about how we perceive relationships in our day to day lives. We also talked about we look at people around us, and how our perceptions of them affect our relationships with them, being a continuation from the first session taken a week ago.

Prof. Dipti Misra presented to us a series of images explaining and dissecting the various relationships that we hold and the factors that make relationships work such as trust in one another and ineptitude and so on. She also discussed how we can eventually attain harmony as a global community and society by following some golden rules.
      
\section*{What I liked about the session}

What I really liked about the session was how the speaker addressed various attributes and parameters that define a relationship, and how they were related to the overall health of the relationship.
    
I also loved how Prof. Dipti used a lesson from the Bhagvad Gita, that states that we must judge people by their intentions and ourselves by our acts, which really resonated a lot with me.

Another aspect of the presentation that struck a chord with me was when Prof. Dipti spoke about the difference between what we aspire to be, and what we currently are.

 \section*{What I disliked about the session}
 
Something that could also have been touched upon more was how to maintain ourselves in a relationship in such a way that would benefit both parties appropriately.

I also felt that the speaker's views on justice were very subjective and that justice isn't always fair even when both parties agree with the verdict.

\section*{Takeaways}
    
I realise how neglectful I have been towards introspecting on the relationships that I have been a part of, and will commit myself to play a more active role in managing my expectations appropriately henceforth.

\section*{Education - Sanskar}

A proper education is very crucial, as the purpose of education is to not only learn about the field you’re interested it, but to get the ability to differentiate between right and wrong, and move society forward, both in a economic/technological and social way.  

Getting an education necessitates a commitment to acquiring concepts and making an attempt to contribute to society's advancement. In the educational process, we must remember that while we should try to offer to society, we should not take it back in any other way.

The purpose of education is to develop the ability to discriminate
between right and wrong, as well as to improve society economically and socially.

\end{document}

\documentclass[a4paper, 12pt]{extarticle}
\usepackage[a4paper, total={6in, 10in}]{geometry}
\usepackage[english]{babel}
\usepackage[utf8x]{inputenc}
\usepackage{amsmath}
\usepackage{graphicx}
\usepackage{indentfirst}

\title{Value Education - 1\\
Knowledge Knots - \uppercase\expandafter{\romannumeral2\relax}}

\author{Sankalp S Bhat\\
2020112018
}
\begin{document}
\maketitle
\thispagestyle{empty}

\section*{About the Session}
Being the second session in a three part series about general knowledge and related things, this session was about how general knowledge is getting talked about less and less by the modern generation, and how the vast depth and intrigue of the subject holds a lot of charm.

Dr. Harjinder "Laltu" Singh presented to us a series of images explaining and dissecting the various facets of general knowledge, namely perception, emotion and reason. Laltu talked about Perception in today's session.
      
\section*{What I liked about the session}

What I really liked about the session was how the speaker addressed various attributes and parameters that define true knowledge by giving an analogy about the five senses
    
I also loved how Laltu used an example holding relevance in current society while he talked about subjects like art or history giving importance to perception, in stark contrast to a definitive subject like science or mathematics.


 \section*{What I disliked about the session}
 
While I fully agree with the speaker for the most part of the presentation, I must say that I couldn't really agree with how the speaker proved to us the existence of god being an atheist.

\section*{Takeaways}
    
I realise that I had to challenge myself constantly to improve myself, and not cage myself and in turn expose myself to the vast depth of knowledge this beautiful world has to offer to us, while also appreciating how I have improved, showing how there is never enough to learn and how not everything I observe is true.


\end{document}

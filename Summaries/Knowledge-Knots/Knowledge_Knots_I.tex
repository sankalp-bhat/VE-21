\documentclass[a4paper, 12pt]{extarticle}
\usepackage[a4paper, total={6in, 10in}]{geometry}
\usepackage[english]{babel}
\usepackage[utf8x]{inputenc}
\usepackage{amsmath}
\usepackage{graphicx}
\usepackage{indentfirst}

\title{Value Education - 1\\
Knowledge Knots - \uppercase\expandafter{\romannumeral1\relax}}

\author{Sankalp S Bhat\\
2020112018
}
\begin{document}
\maketitle
\thispagestyle{empty}

\section*{About the Session}
Being the first session in a three part series about general knowledge and related things, this session was about how general knowledge is getting talked about less and less by the modern generation, and how the vast depth and intrigue of the subject holds a lot of charm.

Dr. Harjinder "Laltu" Singh presented to us a series of images explaining and dissecting the various facets of general knowledge, and started off by presenting to us some media relating to religion and the various morals it imparts.
      
\section*{What I liked about the session}

What I really liked about the session was how the speaker addressed various attributes and parameters that define true knowledge, like uncertainty in expectations and how we had to place ourselves in the world of knowledge.
    
I also loved how Laltu used an example holding relevance in current society in the form of \textit{shlokas}, which emphasized on the importance of knowledge.

An aspect of the presentation that struck a chord with me was when Dr. Harjinder spoke about the injustices in society and how education uplifts the marginalized.

 \section*{What I disliked about the session}
 
While I fully agree with the speaker for the most part of the presentation, I must say that I couldn't really agree with how the speaker proved to us the existence of god, and disliked the lack of in-depth discussion pertaining to the myriad of topics presented to us.

\section*{Takeaways}
    
I realise that I had to challenge myself constantly to improve myself, and not cage myself and in turn expose myself to the vast depth of knowledge this beautiful world has to offer to us, while also appreciating how I have improved, showing how there is never enough to learn.


\end{document}

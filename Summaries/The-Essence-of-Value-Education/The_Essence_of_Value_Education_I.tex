\documentclass[a4paper, 12pt]{extarticle}
\usepackage[a4paper, total={6in, 10in}]{geometry}
\usepackage[english]{babel}
\usepackage[utf8x]{inputenc}
\usepackage{amsmath}
\usepackage{graphicx}
\usepackage{indentfirst}

\title{Value Education - 1\\
The Essence of Value Education - \uppercase\expandafter{\romannumeral1\relax}}

\author{Sankalp S Bhat\\
2020112018
}
\begin{document}
\maketitle
\thispagestyle{empty}

\section*{About the Session}

Being the first session in a two part series about the essence of values and morals in today's society, this session was about the universality and meaning of moral values in society, and its omnipresence.

Dr. Pradeep Ramancharla presented to us a series of images explaining and dissecting the various facets of perception of human values in nature, and about its 4 comprising components. He also went on to discuss the interconnected relationship between the 4 orders, and also later talked in brief about education.
      
\section*{What I liked about the session}

What I really liked about the session was how the speaker addressed various attributes and parameters relating to essentiality of human values from a very unique top-down point of view.
    
I also loved how Dr. Ramancharla used an example holding relevance in current society while he talked about the viciousness of the intense cycle of problems we face, and how we can truly call ourselves separate from animals only when we receive a proper education.

\section*{What I disliked about the session}
 
While I fully agree with the speaker for the most part of the presentation, I must say that I couldn't really agree with how the speaker talked in concrete terms about how money wouldn't affect how happy you are. While I agree money is not everything in life, having no money would lead to a deep depression resulting from a failure to satisfy even the basic human needs.

\section*{Takeaways}
    
I realise that I had to challenge myself constantly to and make proper use of the education that I have received, and have been enlightened about how education is not all that defines a man, we must also put our education in use towards the progress of humanity.

\end{document}

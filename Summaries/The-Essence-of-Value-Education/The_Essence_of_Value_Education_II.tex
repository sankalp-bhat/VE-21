\documentclass[a4paper, 12pt]{extarticle}
\usepackage[a4paper, total={6in, 10in}]{geometry}
\usepackage[english]{babel}
\usepackage[utf8x]{inputenc}
\usepackage{amsmath}
\usepackage{graphicx}
\usepackage{indentfirst}

\title{Value Education - 1\\
The Essence of Value Education - \uppercase\expandafter{\romannumeral2\relax}}

\author{Sankalp S Bhat\\
2020112018
}
\begin{document}
\maketitle
\thispagestyle{empty}

\section*{About the Session}

Being the second and final session in a two part series about the essence of values and morals in today's society, this session was about the universality and meaning of moral values in society, and about our co-existence with the natural orders.

Dr. Pradeep Ramancharla presented to us a series of images explaining and dissecting the various facets of values, the character of human beings, about our harmonious existence with nature and the fundamental propositions to lead a balanced and happy life.
      
\section*{What I liked about the session}

What I really liked about the session was how the speaker addressed various attributes and parameters relating to essentiality of human values and made us introspect on how we carry ourselves on a day to day basis.
    
I also loved how Dr. Ramancharla talked about inner peace, while explaining how it can only be achieved through innate acceptance.

Another aspect of the presentation I liked was how Dr. Ramancharla talked about the gap between our expectations and reality, and about the clarity of our goals and our determination towards achieving them.

\section*{What I disliked about the session}
 
While I fully agree with the speaker for the most part of the presentation, I must say that I couldn't really agree with how the speaker talked in concrete terms about how if one cant lead a harmonious life with our, one cannot harmonise with society. This can clearly be seen in many real life examples of people going through abuse and trauma from families.

\section*{Takeaways}
    
I realise that I had to challenge myself constantly and to have a fixated and determined approach towards my end goal. I also have to learn to live in harmony with society, and not despise it, while also treating everyone equally with compassion.

\end{document}

\documentclass[a4paper, 12pt]{extarticle}
\usepackage[a4paper, total={6in, 10in}]{geometry}
\usepackage[english]{babel}
\usepackage[utf8x]{inputenc}
\usepackage{amsmath}
\usepackage{graphicx}
\usepackage{indentfirst}
\pagenumbering{gobble} 

\title{Value Education - 1\\
\bf{The Imitation Game}}

\author{Sankalp S Bhat\\
2020112018
}
\begin{document}
\maketitle
\thispagestyle{empty}

Alan Turing is part of a team of linguists and mathematicians who have taken on breaking the Nazi code, that could potentially win the war for Britain. Such a plot sounds like it can be cooked purely for a big screen, and sometimes it is hard to remember that it is based on historical events.\\ 

Aside from the war's progress, the story continues with Turing, his relationships with colleagues, and his struggles to accept his identity. Turing's interaction with colleagues can only look a bit like Steve Jobs'esque early in the film. Turing is a typically misunderstood genius that pushes people who do not understand his vision and demeans their work. He underestimates his companions as ignorant and unimaginative, and those around him refuse to understand the importance of the machines that Turing slaves over.\\

But then the tone of the film shifts as a mathematical prodigy, and later Turing’s fiancée, Joan Clarke, forces him to make amends with the members of his code-breaking team. Throughout the film, this tortured genius morphs into a complex character of increasing depth. The reborn Turing finds it challenging to adapt to a dynamic team, with well-written scenes involving classic British banter. The arrogant and stubborn description of Cumberbatch’s Turing gains greater sympathy while the film progresses, as it is found out that his character wrestled with autism and reveals that he was a product of childhood bullying at a preparatory school in England. As the backstory falls in slowly, the audience finds out that the misunderstood genius is emotionally immersive and is no longer a stubborn punch line but a product of disability and trauma.\\

Throughout the film, Turing develops an affinity for the moxie-filled female lead Joan Clarke (Keira Knightley). Clark opposes the world of mid-twentieth-century mathematical men and proves that women can be equal to men. Knightley's story does an excellent job of highlighting this vital message, but it can sometimes get stuck in clichés. In the scene, a male proctor says she is not applying for his secretarial position when Clarke arrives at the supposed "secretarial" exam. At the last moment, just before she was discharged from the test site, Turing delivers a harsh response and hurriedly visited her, asking her if she was working. This kind of anxiety, which can be easily overcome, feels rushed and empty as Clark encounters sexism in the field of mathematics in a short amount of time.\\

Protestant Britain in the mid-20th century was not an ideal place for homosexuality, and  this leads to the end of the film and perhaps the most emphasized social theme, homosexuality. Homosexuality does not lead the plot, but it becomes more pronounced towards the end of the story. Turing is cross-examined because he is gay and knows what the consequences would be if he were discovered. Since the audience is already exposed to Turing’s tragic backstory, this realisation of Turing’s arrest comes with an element of angst and disappointment.\\

Eventually, he is caught for “indecent behaviour”, and this bitter ending ends the legacy of this mathematical genius. The final third of this film is fixated on the struggle of the 20th-century gay community and is therefore highly relevant to the political issues of the 21st century.\\

Homosexuality and gay rights issues are a subject of constant debate, and this message is strongly echoed through the resilience and humanity of the lead character, Sir Alan Turing.\\

The movie is engaging and intriguing, particularly if you go into it knowing nothing at all about Turing or the British endeavours to break the German codes, and the stakes in question. Yet, it is all so dignified and appropriate, and there is a lot more of the story that has not been told in the movie that The Imitation Game (which alludes to a test Turing concocted to decide the complexity of computerized reasoning) wholly satisfies its title, giving us a snapshot of Turing's story, yet never digging profoundly enough to find the real persona behind Alan Turing.


\end{document}

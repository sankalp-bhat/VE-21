\documentclass[a4paper, 12pt]{extarticle}
\usepackage[a4paper, total={6in, 10in}]{geometry}
\usepackage[english]{babel}
\usepackage[utf8x]{inputenc}
\usepackage{amsmath}
\usepackage{graphicx}
\usepackage{indentfirst}
\pagenumbering{gobble} 

\title{Value Education - 1\\
Life is easy, why do we make it so hard?}

\author{Sankalp S Bhat\\
2020112018
}
\begin{document}
\maketitle
\thispagestyle{empty}

\section*{About the Talk}
This TEDx talk was held at Doi Suthep in Thailand, and was delivered by Jon Jandai, an eccentric rice farmer who narrates the events of his life in a very entertaining fashion, ranging from him dropping out of university in Bangkok to returning back to his village to live as a farmer.

Jon Jandai talks about him being born in a poor family, and how conventional dreams of success made him pursue a university education, believing his life will become a lot better in Bangkok. However, he goes through an existential crisis, becoming disgruntled at his current state in an overcrowded city, where he had to work many times harder at subjects that he felt were 'destructive'.

He eventually returns to his village and realises that a happy life doesn't require much effort, and he can lead a fulfilling life cultivating rice and building his own house. As he meets all his basic needs without much effort, he can lead a satisfying life by putting in just 2 months of work, while resting easy for the other 10 months. He drives home some compelling arguments about humans in the modern era unnecessarily complicating things for themselves, but I believe he is looking at the world through rose tinted glasses, and firmly disagree with some of his arguments. 
      
\section*{What I agreed with the speaker on}

I agree with Jon about the pitfalls of the ever increasing materialism among humans, and fully believe that material things don't constitute the person inside.

He gives a brilliant example about how he stopped buying jeans, after saving money to buy them and realising that it doesn't change who he is as a human. He also goes on to mention about his friend at university who achieved the highest grades, but is still displeased in life and has mortgages on his house, whereas Jon built his house himself with minimal effort and none of the worries of loans. By chasing after material things, people don't realise that true inner happiness cant be attained by physical objects, but inner bliss comes from partaking in something that you truly enjoy. 

Jon drives home this point brilliantly by mentioning that he cultivates rice on his piece of land, and has no need to buy a new clothes often, while being satisfied with what he has.

He also mentions about how modern humans are struggling to make the basic needs of food, clothing and shelter available to all, and hence mankind is devolving and become more uncivilized by the day.
     
 \section*{What I disagreed with the speaker on}
 
While I agree with Jon about the ever growing materialism in this world, there was plenty i disagreed with as well. 

He looks down upon a modern college education as 'boring' and 'destructive', he forgets to mention the overall good that an invention the likes of modern medicine has brought to this planet. While he says that he can get rid of basic sickness with the use of water, we can clearly see that such a simple treatment would fail to cure dangerous and life-threatening diseases like cancer. 

The spread of widespread university education has also resulted in overall advancement of the morality of man, with the spread of progressive values marking the end of dangerous ideas, that harm the very social fabric of man. Without widespread education, we would've probably been living in a society where evils like slavery and discrimination would be normalised. 

\section*{Conclusion}

Jon Jandai raises some good points about humans making life more complicated for themselves, and making it more difficult than it needs to be, but I believe that he talks from a very optimistic point of view, and it is impractical for everyone to follow his advice given the circumstances that they are born in.

\end{document}

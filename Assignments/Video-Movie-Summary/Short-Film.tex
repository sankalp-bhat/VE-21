\documentclass[a4paper, 12pt]{extarticle}
\usepackage[a4paper, total={6in, 10in}]{geometry}
\usepackage[english]{babel}
\usepackage[utf8x]{inputenc}
\usepackage{amsmath}
\usepackage{graphicx}
\usepackage{indentfirst}
\pagenumbering{gobble} 

\title{Value Education - 1\\
Right Here, Right Now - A Review}

\author{Sankalp S Bhat\\
2020112018
}
\begin{document}
\maketitle
\thispagestyle{empty}

\section*{About the Production}
This short film comprises of 2 parts, and is set in Mumbai, India. Produced by Anand Gandhi, an independent filmmaker based in Mumbai, the 2 parts which are each approximately 15 minutes long portray how the things we do affect others through a chain of events, something akin to the 'Butterfly Effect'.

The short film was aired at the British-Indo film festival, and has since gained critical acclaim for its unique production style and the powerful message which it delivers.

The production is also filmed in such a way as to appeal to the common man, produced in a simple and relatable setting.

Aside from its striking concept, the film also highlights the richness and variety of Indian culture, by showcasing segments filmed in 8 different Indian languages, all in the span of 29 minutes.

\section*{Part 1}

In the first part of the production, we observe an interesting chain of events, where a young man being rude to his mother and refusing to eat his breakfast, eventually leading into him getting involved in an accident and the culprit getting away. The director portrays this chain of events in a very natural way and we get indulged in the movie without questioning the likelihood of such a chain of events happening in reality.

From the first part, we observe that despite the possibility of the chain of negativity breaking at any link, nobody makes an attempt to be positive, and hence ended up furthering the negativity, eventually leading to a disaster. We take home the message that a negative action, despite how small it might be, has the potential to cause a huge impact and even potentially bite us back.

We must make a concerted effort to avoid negativity, and also realise that venting our rage out at someone is only gonna cause harm to them.
     
 \section*{Part 2}
 
In the second part of the production, we observe the polar opposite of the outcomes of the first part, wherein a small compliment goes a long way, leading to a chain of events ending with a man being helped out of an accident where he was severely injured.

Even though the action of the man complimenting his younger brother's art took little to no effort, it ended up helping a man in desperate need. We should always try our level best to spread as much positivity in this world as possible.

\section*{Conclusion}

The movie really drives home the point of our actions having consequences, and the severity of the outcomes and the people affected by them. This movie has helped shed light on the fact that I must take responsibility for my actions and not take out my anger on someone who's not deserving of it.

\end{document}
